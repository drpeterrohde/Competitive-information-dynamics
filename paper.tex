\documentclass[aps,rmp,twocolumn,amsmath,amssymb,nofootinbib,superscriptaddress]{revtex4}

\newcommand{\bra}[1]{\langle#1|}
\newcommand{\ket}[1]{|#1\rangle}
\newcommand{\op}[2]{\hat{\textbf{#1}}_{#2}}
\newcommand{\dagop}[2]{\hat{\textbf{#1}}_{#2}^\dag}
\usepackage[pdftex]{graphicx}
\usepackage{mathrsfs}
\usepackage[colorlinks]{hyperref}

\begin{document}

\bibliographystyle{apsrev}

%
% Title
%

\title{The theory of competitive traits}

%
% Authors
%

\author{Peter P. Rohde}
\email[]{dr.rohde@gmail.com}
\homepage{http://www.peterrohde.org}
\affiliation{Centre for Quantum Computation and Intelligent Systems (QCIS), Faculty of Engineering \& Information Technology, University of Technology Sydney, NSW 2007, Australia}

\date{\today}

\frenchspacing

%
% Abstract
%

\begin{abstract}
\end{abstract}

\maketitle

\section{Definitions}

Strategies are associated with a trait, not with the group itself.

Trait carries algorithmic information content.

Strategy is defined as being between a trait group and its compliment. Gives rise to indirect competition.

Trait competitiveness, $C_t$: score for extent to which a group member $g\in G_t$ abiding by the trait group's strategy $S_{G_t}$ enhances the survival rate of trait $t$.

Discuss competing vs non-competing traits. Local strategy optimisation globally optimises.

There is competition between non-complimentary trait group, i.e partially overlapping. Explains cooperation between groups with overlapping interests.

Traits are ratio of expression, not just binary

An individual's tendency to obey a strategy: $T_{g,S}$

Inter-trait enhancement. $t_1$ ehnahnces expression of $t_2$ - cooperation

Indirect competition: piggbacking off the strategy employed by a correlated trait.

If traits are perfectly correlated, their respective trait groups are necessarily identical, $G_{t_1}=G_{t_2}$. Nonetheless, the strategies associated with the traits may be distinct, $\mathcal{S}(t_1)\neq\mathcal{S}(t_2)$.

\subsection{Traits}

A \textit{trait} is an entirely arbitrary characteristic that may be associated with an individual. It may be genotypic, phenotypic, memetic, cultural, hereditary or non-hereditary, acquired or inate, physical or psychological -- it may be defined arbitrarily and abstractly as any feature by which an individual or group can be associated.

We use $t$ to denote a single trait, or $\vec t=\{t_1,\dots,t_n\}$ to denote a set of traits.

\subsubsection{Correlated traits}

\subsubsection{Competitive vs non-competitive traits}

Selective pressures. Take two perfectly correlated traits. One may act competitively, the other not. The later inherits all benefits of the former, without actively pursuing a coordinated strategy of its own. Indirectly competitive.

\subsection{Individuals \& groups}

We use $U$ to denote the \textit{universe} -- the set of all individuals participating in the model under consideration. An \textit{individual} is a single participant within the universe, characterised entirely by its complete set of traits,
\begin{align}
	g=\{t_1,\dots,t_n\} \in U.
\end{align}

A \textit{group} is a set of individuals,
\begin{align}
	G = \{g_1,\dots,g_n\}.	
\end{align}

In the case of \textit{acquired traits}, an individual's trait set is dynamic, undergoing the update rule,
\begin{align}
	g\to g\cup t,	
\end{align}
where $g$ acquires trait $t$.

A \textit{trait group} is the set of all individuals exhibiting a trait or set of traits,
\begin{align}
	G_t = \bigcup_i g_i \,|\, t\in g_i.
\end{align}
The complimentary non-trait group is denoted $\bar G_t$. Together the satisfy the completeness relation,
\begin{align}
G_t \cup \bar G_t = U,	
\end{align}
since traits are binary.

\subsection{Strategies}

A \textit{strategy} is defined in a general multi-player game-theoretic context as the set of algorithms employed between a set of \textit{players}, $p_i$. A player may be an individual or a group of individuals. Thus, strategies may be defined at an individual level (\textit{individual strategies}), or collectively at a group level (\textit{group strategies}).

The joint strategy employed by $n$ players $p_i$ against $m$ players $q_j$ is denoted,
\begin{align}
	\mathcal{S}(p_1,\dots,p_n|q_1,\dots,q_m).
\end{align}
The special case of $n=1$ refers to a single player engaging in a strategy independently of other players (i.e not a joint strategy). A we adopt the notation $\mathcal{S}'$ for an opposing strategy,
\begin{align}
	\mathcal{S}(p_1,\dots,p_n|q_1,\dots,q_m) = \mathcal{S}'(q_1,\dots,q_n|p_1,\dots,p_m).
\end{align}

Note that individuals within a group may not behave homogenously, thus the strategy employed by individuals within a group may in general be distinct from their collective group strategy, which described by the cumulative action of all participating individuals. Thus, in general,
\begin{align}
	\mathcal{S}(g\in G|q)\neq \mathcal{S}(G|q).
\end{align}

\subsubsection{Indirect strategies}

A trait might directly pursue its own competitive strategy, or indirectly inherit it via correlated traits doing so. For example, a genetically induced trait might be highly competitive, but the responsible genes additionally give rise to other secondary traits that are not inherently competitive in their own right. In this instance, the secondary correlated traits indirectly inherit competitiveness from the more competitive primary trait.

\subsection{Payoff}

The \textit{payoff} of a strategy for entity $x$, $\mathcal{P}_x(\mathcal{S})$, is the expectation value of the `reward' (in arbitrary units of utility) obtained by executing a strategy. The payoff may be positive (for `successful' strategies), or negative (for `unsuccessful' ones).

For a trait $t$ to develop a collective strategies to guarantee its propagation, their objective is to maximise the joint payoff of the respective trait group, $G_t$

A special case arises when players compete for finite resources, in which case the sum of their respective payoffs is bounded by the resources' abundancy $c$,
\begin{align}
	\mathcal{P}(\mathcal{S}) + \mathcal{P}(\mathcal{S}') \leq c.
\end{align}

Another specialised case is \textit{zero-sum games} in which players' payoffs are exactly opposite, i.e gains made by one come at the direct expense of the other,
\begin{align}
\mathcal{P}(\mathcal{S}) = -\mathcal{P}(\mathcal{S}').
\end{align}

\subsubsection{Individual payoff}

\subsubsection{Group payoff}

The payoff for a group is simply given by the sum of the payoffs benefitting the group by all individuals in the universe,
\begin{align}
	\mathcal{P}_{G}(\mathcal{S}(G)) = \sum_{g\in U} \mathcal{P}_G(g)
\end{align}

In the case of a non-competing trait that achieves payoff indirectly via its correlation with competing traits, we can observe properties of the form,
\begin{align}
	\mathcal{P}_{G_1}(\emptyset) \propto \mathcal{P}_{G_2}(\mathcal{S})
\end{align}
That is, the group $G_1$ benefits passively (implementing no strategy of its own) via the payoff bestowed upon group $G_2$ pursuing its own strategy.

\section{Selfish vs altruistic behaviour}

Conditions where pursuing self-interest is congruent with group interest.

\end{document}
