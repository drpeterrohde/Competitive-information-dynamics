\documentclass[twocolumn, aps, rmp, amsmath, amssymb, nofootinbib, superscriptaddress, longbibliography, floatfix, table-of-contents, eqsecnum]{revtex4-1}

\usepackage[pdftex]{graphicx}
\usepackage[usenames, dvipsnames, svgnames, table]{xcolor}
\usepackage{mathrsfs}
\usepackage[colorlinks, breaklinks]{hyperref}
\usepackage{url}
%\usepackage{colortbl}
\usepackage[section]{placeins}
%\usepackage{array}
\usepackage{amsmath}
%\usepackage[noend]{algpseudocode}
%\usepackage{mdframed}
\usepackage{type1cm}
\usepackage{lettrine}
\usepackage{afterpage}
\usepackage[english]{babel}
\usepackage{lmodern}
\usepackage{multirow}
\usepackage[margin=0pt, font=small, labelfont=bf, labelsep=endash, justification=centerlast, labelsep=colon]{caption}
\usepackage{microtype}
\usepackage{booktabs}
\usepackage{fancyhdr}
%\usepackage{braket}

\hypersetup{
	pdfauthor = {Peter P. Rohde},
	pdftitle = {Competitive information dynamics - The theory of competitive traits & meta-strategic game theory},
	pdfsubject = {Competitive information dynamics}
}

\newcommand{\famousquote}[2]{{\textit{``#1''} --- #2.}\newline}

\begin{document}

%
% Title
%

\title{Competitive information dynamics --- The theory of competitive traits \& meta-strategic game theory}

%
% Authors
%

\author{Peter P. Rohde}
\email[]{dr.rohde@gmail.com}
\homepage{http://www.peterrohde.org}
\affiliation{Centre for Quantum Software \& Information (UTS:QSI), University of Technology Sydney, Australia}

\date{\today}

\frenchspacing

%
% Abstract
%

\begin{abstract}
\end{abstract}

\maketitle

\tableofcontents

\section{Notes}

group leaders, worship

Trait abstractly carries algorithmic information content, not attached to a particular physical medium.

Strategy is defined as being between a trait group and its compliment. Gives rise to indirect competition.

Trait competitiveness, $C_t$: score for extent to which a group member $g\in G_t$ abiding by the trait group's strategy $S_{G_t}$ enhances the survival rate of trait $t$.

There is competition between non-complimentary trait group, i.e partially overlapping. Explains cooperation between groups with overlapping interests.

Traits are ratio of expression, not just binary

An individual's tendency to obey a strategy: $T_{g,S}$

Inter-trait enhancement. $s_1$ ehnahnces expression of $s_2$ - cooperation

If traits are perfectly correlated, their respective trait groups are necessarily identical, $G_{s_1}=G_{s_2}$. Nonetheless, the strategies associated with the traits may be distinct, $\mathcal{S}(s_1)\neq\mathcal{S}(s_2)$.

self-competition for strategically comaptible goals:
for distinct strategies for different traits, where the traits are highly correlated or subset, better strategy is for the weaker trait to piggyback off the stronger common trait than to ouruse its own. strong supergroup strategy leads to weaker subgroup strategy to yield pririty to the supergroup strategy.

trait absirbtion by strategies advancing correlated traits. 

section: medium for information. incl. collective.

tit for tat is meta-strategically stable, even though no stability for groups employing it.

\begin{table}[!htbp]
\begin{tabular}{|c|c|c|c|c|c|}
\hline
Player               &          & \multicolumn{3}{c|}{$X$}       &      \\ \hline
                     & Strategy & $S_1$    & $S_2$    & $S_3$    & Meta \\ \hline
\multirow{3}{*}{$Y$} & $S_1$    & $(0,0)$  & $(1,-1)$ & $(2,-2)$ &      \\ \cline{2-6} 
                     & $S_2$    & $(-1,1)$ & $(0,0)$  & $(1,-1)$ &      \\ \cline{2-6} 
                     & $S_3$    & $(-2,2)$ & $(-1,1)$ & $(0,0)$  &      \\ \hline
                     & Meta     &          &          &          &      \\ \hline
\end{tabular}
\end{table}

%
% Mathematical background
%

\part{Mathematical background}

\section{Traits}

A \textit{trait} is an entirely arbitrary characteristic that may be associated with an individual. It may be genotypic, phenotypic, memetic, cultural, hereditary or non-hereditary, acquired or innate, immutable or mutable, physical, psychological, or entirely abstract in nature -- it may be defined arbitrarily and abstractly as any feature by which individuals can be identified or associated with as having in common.

We use $s$ to denote a single generic trait, or $\vec s=\{s_1,\dots,s_n\}$ to denote a set of traits.

\subsection{Correlated traits}

In general, different traits do not express themselves as independent random variables. They often exhibit all manner of correlations, even between traits defined across different mediums. For example, a direct correlation between an inherited physical trait (such as skin colour, largely genetically encoded), and a non-inherited, associative one (such as political party affiliation, which is entirely psychologically motivated, and subsequently reinforced via the strategic benefit arising from the associated trait group strategy).

Traits emerge via many distinct mechanisms, often with evolutionary origins. The evolutionary nature of the emergence and propagation of many traits implies that some will share common historical origins. A shared evolutionary history between very distinctly expressed traits may leave residual correlations in their expression at much later stages -- a sort of `evolutionary hidden variable theory', to utterly abuse the vernacular of modern physicists, if you will.

In general, we expect all manner of correlations between distinct traits. Within a given group (or the universe as a whole), we can quantify this via the covariance in their expression within the group,
\begin{align}
C_G(s_1,s_2) = \text{cov}(s_1\in G | s_2\in G).
\end{align}

\subsection{Trait heirarchies}

A special case in which trait correlation arises, is when traits manifest themselves in a hierarchical manner, whereby a set of trait groups form sub-groups of one another,
\begin{align}
G_{s_1} \subset G_{s_2} \subset \dots \subset G_{s_n}.
\end{align}
In such a sequence, exhibiting trait $i$ necessarily implies also exhibiting trait $j$, for \mbox{$i<j$}. Such sequences could also bifurcate into tree-like group inclusion hierarchies.

\section{Welcome to the battle arena}

We use $U$ to denote the \textit{universe} -- the set of all individuals participating in the competitive arena under consideration. Mathematically, it is a self-contained space, and does not interact with any externalities. However, new individuals may enter the universe upon creation, or exit upon destruction. Thus, the universe is a dynamic system, and although it exists continuously, it may evolve across orthogonal states over time.

An \textit{individual} is a single participant within the universe, characterised entirely by its complete trait set,
\begin{align}
	g &= \{s_1,\dots,s_n\},\nonumber\\
	g &\in U.
\end{align}

A \textit{group} is simply a set of individuals,
\begin{align}
	G = \{g_1,\dots,g_n\}.	
\end{align}
The importance of defining groups as distinct mathematical objects in their own right, is that they may exhibit collective behaviour, distinct from the individual behaviour of their constituents.

A \textit{trait group} is the set of all individuals exhibiting a trait or set of traits in common,
\begin{align}
	G_t = \bigcup_{i\in U} g_i \,|\, t\in g_i.
\end{align}
The complimentary \textit{non-trait group} is denoted $\bar G_t$. Together they satisfy the set relations,
\begin{align}
G_t \cup \bar G_t &= U \,\,\text{(completeness)},\nonumber\\
G_t \cap \bar G_t &= \emptyset \,\,\text{(exclusive)},
\end{align}
since traits are defined as binary.

\section{Strategies}

A \textit{strategy} is defined in a game-theoretic context as the algorithm employed by one player when interacting with another to advance its agenda. A player may be an individual or a group. Thus, strategies may be defined at an individual level (\textit{individual strategies}), or collectively at a group level (\textit{group strategies}).

We denote the strategy employed by $p$ against $q$ as,
\begin{align}
	\mathcal{S}(p|q).	
\end{align}
$\mathcal{S}'$ denotes the opposing strategy in the reverse direction,
\begin{align}
	\mathcal{S}(p|q) = \mathcal{S}'(q|p).
\end{align}
Two opposing strategies may be chosen entirely independently -- the players could be engaging symmetrically using the same strategy, or asymmetrically using different ones.

%The joint strategy employed by $n$ players $p_i$ against $m$ players $q_j$ is denoted,
%\begin{align}
%	\mathcal{S}(p_1,\dots,p_n|q_1,\dots,q_m).
%\end{align}
%The special case of $n=1$ refers to a single player engaging in a strategy independently of other players (i.e not a joint strategy).

Individuals within a group may not behave homogenously or independently. The strategy employed by individuals from a group when acting in isolation may in general be distinct from their collective group strategy. A group strategy may arise as an emergent phenomenon, entirely distinct in nature from the behaviour of the individuals within it. It could also arise as a collective feature of competition between subgroups within the group, resulting in collective dynamics. Thus, in general,
\begin{align}
	\mathcal{S}(g\in G|q)\neq \mathcal{S}(G|q).
\end{align}

\subsection{Strategies as information}

A strategy is not inherently physical or attached to a particular medium. Rather, it is to be considered abstractly in an algorithmic sense, purely informational in nature, that could be encoded into any medium or executed upon any sufficient computational platform. The players employing a strategy can be regarded as \textit{hosts} for the strategy, which encode, execute, and propagate it to advance their interests.

%$I(\mathcal{S})$

\section{Payoffs}

The \textit{payoff} of a strategy for player $p$, $\mathcal{P}_p(\mathcal{S})$, is the expectation of `reward' (in arbitrary units of utility) obtained by executing a strategy. The payoff may be positive (for `successful' strategies advancing the player's interests), or negative (for `unsuccessful' ones).

For a trait $s$ to develop a collective strategy for advancing its propagation, the objective of a successful strategy is to maximise the collective payoff for the respective trait group, $G_s$.

A special case arises when players compete for finite resources, in which case the sum of their respective payoffs is bounded by the resources' abundancy $c$,
\begin{align}
	\mathcal{P}_p(\mathcal{S}_p) + \mathcal{P}_q(\mathcal{S}_q) \leq c.
\end{align}

\textit{Zero-sum games} arise when $c=0$, in which case players' payoffs are exactly opposite, i.e gains made by one come at the direct expense of the other,
\begin{align}
\mathcal{P}_p(\mathcal{S}_p) = -\mathcal{P}_q(\mathcal{S}_q).
\end{align}

Alternately, opposing strategies may collectively act towards their mutual benefit -- a symbiotic strategy,
\begin{align}
\mathcal{P}_p(\mathcal{S}_{p\cup q}) > \mathcal{P}_p(\mathcal{S}_p\cup \mathcal{S}_q).
\end{align}
Here players effectively engage in a cooperative inter-group strategy, collectively enhancing one another.

\subsection{Individual payoff}

\subsection{Group payoff}

The payoff for a group is simply given by the sum of the payoffs benefitting the group by all individuals in the universe,
\begin{align}
	\mathcal{P}_{G}(\mathcal{S}(G)) = \sum_{g\in U} \mathcal{P}_G(g)
\end{align}

A non-competing trait may benefit indirectly via its correlation with competing traits, where we observe properties of the form,
\begin{align}
	\mathcal{P}_{G_1}(\emptyset) \propto \mathcal{P}_{G_2}(\mathcal{S})
\end{align}
That is, the group $G_1$ benefits passively (implementing no strategy of its own) via the payoff bestowed upon group $G_2$ pursuing its own strategy.

%
% The theory of competitive traits
%

\part{The theory of competitive traits}

\section{Intrinsic vs acquired traits}

In the case of \textit{acquired traits}, an individual's trait set is dynamic, undergoing the update rule,
\begin{align}
	g\to g\cup t,	
\end{align}
where $g$ acquires trait $t$.

\section{Indirect strategies}

A trait might directly pursue its own competitive strategy, or indirectly inherit it via correlated traits doing so. For example, a genetically induced trait might be highly competitive, but the responsible genes additionally give rise to other secondary traits that are not inherently competitive in their own right. In this instance, the secondary correlated traits indirectly inherit competitiveness from the more competitive primary trait.

\section{Associative traits}

trait is a group phenomenon, not an intrinsically individual one. the individual acquires the trait upon association with the group. e.g `group identity'. This is in contrast to hereditary traits, which in general are intrinsic upon creation and immutable in nature.

\section{Adoptive traits}

preprioritising strategies to promote another trait that has higher success.

\section{Trait substitution}

group has multiple traits in common. if the strategy swaps between them it makes no difference mathematically. e.g if Gs1 approx Gs2, then if the strategy advancing Gs1 switches over to advancing Gs2, is algorithmically equivalent and can be regarded as interchangeable.

\section{Strategic subservience}

\famousquote{Madam Speaker, I yield to the honourable gentleman.}{Senator Marmaduke, PhD}

self-competition for strategically comaptible goals:
for distinct strategies for different traits, where the traits are highly correlated or subset, better strategy is for the weaker trait to piggyback off the stronger common trait than to ouruse its own. strong supergroup strategy leads to weaker subgroup strategy to yield pririty to the supergroup strategy.

\begin{align}
\mathcal{P}_{s_1}(\mathcal{S}_{s_2}) > \mathcal{P}_{s_1}(\mathcal{S}_{s_1}), \,\,s_1\subset s_2.
\end{align}

\section{Selfish vs altruistic behaviour}

Conditions where pursuing self-interest is congruent with group interest.

Altruism,
\begin{align}
\mathcal{P}_G(\mathcal{S}_g) > \mathcal{P}_g(\mathcal{S}_g),
\end{align}
where
\begin{align}
	g \subset G.
\end{align}

Altruism with expectation for long-term reward. Not actually altruistic, rather self-serving with foresight.

Altruism towards others without expectation for any reward whatsoever -- submissive to the greater good of the overarching group. True altruism in the sense of self-sacrifice.

\section{Competitive vs non-competitive traits}

Selective pressures. Take two perfectly correlated traits. One may act competitively, the other not. The later inherits all benefits of the former, without actively pursuing a coordinated strategy of its own.

The purpose of a trait group engaging in a group strategy is to advance the survival of the trait defining the group. However, it is to be expected that some trait groups will not naturally gravitate towards deploying survival strategies. Specifically, if the occurrence of a trait amongst individuals is inherently strategy-independent, there is no incentive to deploy one. For example, a physical trait that is not heritable, but rather emerges entirely randomly amongst individuals, naturally fits this description.

In such cases we say that the trait group is strategy neutral, denoted,
\begin{align}
	\mathcal{S}=\emptyset,
\end{align}
in which case the trait group does not actively pursue any form of collective strategy against competitors.

%
% Meta-strategy
%

\part{Meta-strategy}

\section{Meta-strategic game theory}

Within the mathematical framework we have introduced, traits may be defined \textit{completely arbitrary} as characteristics exhibited by individuals, with which they can subsequently be associated with the respective trait groups. Since a trait is very much entitled to defined as a behavioural one (or even as an acquired behavioural one), we can legitimately define traits according to the strategies employed by the entity. We refer to groups defined according to their employed strategies as \textit{strategy groups}, where the strategy itself is the trait.

Considering the game-theoretic dynamics between strategy groups can now be considered entirely equivalent to competition between strategies themselves, in the abstract sense where a strategy is purely an information unit competing to advance itself. Now the strategy itself is competing for survival, rather than merely being a tool employed for the advancement of a group. Since many distinct groups may be competing in accordance with the same strategy, its survival needn't be stipulated by a single group employing it, but rather collectively by all groups employing it -- the strategy group. Groups employing a strategy may simply be regarded as \textit{hosts} for the advancement of the strategy. The means of implementation and transmission of a strategy needn't be held in common within the strategy group. All that determines the survival of the strategy group is the abstract algorithmic representation of the strategy, not the medium for storing and executing it, much as a software library can be deployed across multiple platforms and contribute to any number of distinct applications, or identical legal systems implemented in any number of countries.

Alternately, individuals and groups constitute nothing more than a medium in which strategies reside and compete. Neither individuals, groups, nor traits compete. Rather, competition takes place entirely in an abstract information-theoretic sense, where non-physical units of information compete for survival, exploiting whatever physical hosts in which they are able to encode and execute themselves.

Strategic traits: the trait of employing a strategy, $s=\mathcal{S}$. Payoff for strategic trait group is net survival of strategy itself, not a particular group employing the strategy. Strategy is aiming to advance itself, using groups employing it as hosts for itself. May not care about a specific group, and willing to sacrifice them so long as the strategy is preserved long-term. In the case of zero-sum games, might act indifferently.

\subsection{Strategy groups}

Notion of meta-strategy: actual goal of a strategy is to advance the strategy group, not an individual group employing the strategy. When the strategy spreads across multiple groups which evolve independently, it undergoes evolution at the meta-level. Now the strategy is purely informational, and is no longer limited to a particular host residing in a particular medium. It can switch across mediums and evolve as such. If groups employing old mediums die out, the strategy survives via implementation in the new medium.

Strategies don't inherently work for the groups they represent -- the groups are merely a medium, hosts adopted by the strategies to advance their own propagation. In some instances these may be congruent. In others they are not, and sacrificing a group is inconsequential or to the advantage of the perpetuation of the strategy.

Describe new game-theoretic model where utility for strategy is utility of all who employ the strategy, rather than a single group who does.

a group employing a strategy may suffer negative payoff upon another group adopting the same strategy. however the meta-utility of the strategy is positive in this case.

\section{Meta-utility}

\begin{align}
\mathcal{P}_{\mathcal{S}_i} = \sum_{t} \mathcal{P}_{G_t}(\mathcal{S}_i)\, |\, \mathcal{S}_{G_t} = \mathcal{S}_i.
\end{align}
For all groups employing $\mathcal{S}_i$. Assuming the existence of multiple, distinct (not perfectly correlated) trait groups, this necessarily implies double-counting of individuals. However, this poses no conceptual problem, since these multiple counts by design reflect the reproductive strength of the strategy, not the individuals it represents, on the basis that individuals employing the same strategy on multiple fronts similarly have multiple independent avenues by which to propagate it.

when a strategy is employed by just a single player, the meta-utility of the strategy equates to the utility to the player.

when a strategy is employed by two competing groups engaging in a zero-sum game, the utility to the strategy is invariant under the inter-group game outcomes. strategic meta-utility is therefore highly stable and robust.

in general, for positive-sum games with players employing the same strategy, the meta-utility to the strategy is positive, independent of outcomes of individual players.

\section{Meta-strategic game theory}

\begin{align}
\mathcal{P}_{\mathcal{S}_1} > \mathcal{P}_{\mathcal{S}_2}	
\end{align}
even when
\begin{align}
\mathcal{P}_1(\mathcal{S}_1) < \mathcal{P}_1(\mathcal{S}_2)
\end{align}

show game payoff matrix. show meta-strategy payoff matrix.

Find examples of contradictions between the two.

A strategy can exhibit meta-strategic stability in the absence of any stability amongst those employing it.

When a group abandons an old strategy and implements a new one it undergoes trait substitution at the strategic level. The strategy itself has lost, even though its former host survives. A new winning strategy advances the cause of the group adopting it, expanding its own influence in the process.

%
% Quantum information - A new era for strategic dominance
%

\part{Quantum information -- A new era for strategic dominance}

Future quantum strategies. Transition in computational complexity of strategies.

Finding optimal strategies is \textbf{EXP}-complete in general. We can only implement \textbf{BPP}. But in future can implement \textbf{BQP}. Also quadratically enhance \textbf{NP}-complete. In relativistic setting possibly extend to \textbf{postBQP}, \textbf{\#P}.

%
% Humanism
%

\part{Humanism}

\section{The origins of universal human values}

e.g tit for tat acroass species

rir for tat as foundrion for human religion and morals.

\section{Tribalism -- Ideology, religion \& the era of identity politics}

\section{The contradictions \& viability of mankind in the face of human intelligence}

Philosophical: for humanity, our group strategies are successful at promoting the strategy. Unsuccessful at promoting the groups. We must realign our strategies to become congruent with the trait group `humanity'. Enhance human interests, rather than the strategy. This may require making our old strategies extinct, and artificially replacing them with a new meta-group strategy. We can do this using our capacity for intelligence, computation, reason, and morality.

must optimise our strategies, rather than risk being exploited and bled dry by the prevailing meta-strategy, which has until now dictated them.

Must optimise the utility to ourselves\footnote{Militant vegans may instead advocate optimising $\mathcal{P}_{G_\text{pandas}}(\mathcal{S}_\text{mankind})$, whereas Jains may opt for optimising $\mathcal{P}_{G_\text{beings}}(\mathcal{S}_\text{mankind})$.},
\begin{align}
	\mathcal{P}_{G_\text{mankind}}(\mathcal{S}_\text{mankind}),
\end{align}
rather than capitulate to optimising,
\begin{align}
	\mathcal{P}_{\mathcal{S}_\text{mankind}}(\mathcal{S}_\text{mankind}),
\end{align}
knowing well that $\mathcal{S}_\text{mankind}$ overlaps enormously with the strategies employed by many other species, in which case our extinction may be inconsequential or to the benefit of the prevailing meta-strategy.

Want,
\begin{align}
\mathcal{P}_{G_\text{mankind}}(\mathcal{S}_{G_t}) \geq \mathcal{P}_{G_t}(\mathcal{S}_{G_t})\,\,\forall\, t
\end{align}

\section{Can mankind strategically evolve?}

mankind's strategy appears to be a mixture between \textsc{Fairness} and \textsc{Dominance}, depending on the degree of asymmetry between players. cite psych studies supporting this.

Variation of tit-for-tat is genetically hard-coded into us. Expressed widely across many competitive individual and group scenarios that we engage in.

Tit-for-tat is meta-strategically stable. Little meta-strategic incentive to evolve.

Distinction between technological improvement and strategic evolution.

\begin{align}
\mathcal{S}_\textsc{Tit-For-Tat} \approx \mathcal{S}_\textsc{Fairness}	
\end{align}

\textsc{Tit-For-Tat} evolves \textsc{Fairness}, in long-time limit results in \textsc{Fair-Dominance} via \textit{preferential attachment} occurring as a result of bargaining asymmetry. Leads to Pareto distribution,
\begin{align}
f_\text{Pareto}(x) = \left\{\begin{array}{ll}
\frac{\alpha x_m^\alpha}{x^{\alpha+1}}, & x\geq x_m,\\
0, & x<x_m.
\end{array}\right.
\end{align}
$x_m\geq 0$ is minimum value for $x$ (\textit{scale} parameter), $\alpha>1$ is the \textit{shape} or \textit{Pareto index}, which characterises the degree of asymmetry in the distribution.

\begin{align}
	\mathcal{S}_\textsc{Fair-Dominance} \subset \mathcal{S}_\textsc{Fairness}
\end{align}
therefore meta-strategically stable.

\begin{align}
	\lim_{t\to\infty} \mathcal{S}_\textsc{Fairness} \to \mathcal{S}_\textsc{Fair-Dominance}
\end{align}

But
\begin{align}
	\mathcal{S}_\textsc{Fair-Dominance} \approx \mathcal{S}_\textsc{Dominance}
\end{align}

Exponential growth,
\begin{align}
W(t) = \beta e^{\gamma t}	
\end{align}

Mean wealth is,
\begin{align}
	{\overline W}(t) &= E[f_\text{Pareto}(x)] \cdot W(t) \nonumber\\
	&= \frac{\alpha \beta x_m e^{\gamma t}}{\alpha - 1}
\end{align}

We require ${\overline W}(t+1)\geq {\overline W}(t)$. In the long-time limit when $\gamma\to 0$, this implies the Pareto index become stable $\alpha_{t+1} = \alpha_t$, otherwise mean payoff becomes negative and the strategy becomes unstable for the group.

So long as this condition is maintained, we have non-negative mean payoff at each time step.

So long as both individual and group payoffs remains positive, the strategy disincentivises opposition, and remains strategically stable. Under exponential growth conditions, this can be maintained, will tend to be the case for the sake of the strategy remaining viable. When growth conditions plateau (which inevitably they must under finite resource constraints, i.e $\gamma\to 0$), if the Pareto index continues to increase $\alpha(t+1)>\alpha(t)$, which is a result of existing asymmetry, not continued growth, this necessarily implies reduced mean payoff,
\begin{align}
	E[f_\text{Pareto}(x)] = \frac{\alpha x_m}{\alpha - 1},
\end{align}
implying an increasing subset of individuals whose individual payoff turns negative, $\mathcal{P}_\text{g}(t+1)<\mathcal{P}_\text{g}(t)$, at which point the strategy becomes increasingly unviable for the group since the outcome violates the perception of fairness. However, although strategically unstable at the individual level, it remains meta-strategically stable. No meta-strategic incentive for rectification.

\bibliography{bibliography}

\end{document}
